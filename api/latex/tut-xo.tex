\hypertarget{tut-xo_Contents}{}\section{Contents}\label{tut-xo_Contents}

\begin{DoxyItemize}
\item \hyperlink{tut-xo_launch}{Launch Xtal\+Opt}
\item \hyperlink{tut-xo_init}{Enter composition and restraints}
\item \hyperlink{tut-xo_opt}{Optimizer setup}
\begin{DoxyItemize}
\item \hyperlink{tut-xo_vasp-opt}{V\+A\+S\+P}
\item \hyperlink{tut-xo_gulp-opt}{G\+U\+L\+P}
\item \hyperlink{tut-xo_pwscf-opt}{P\+Wscf}
\item \hyperlink{tut-xo_castep-opt}{C\+A\+S\+T\+E\+P}
\end{DoxyItemize}
\item \hyperlink{tut-xo_qisetup}{Queue setup}
\begin{DoxyItemize}
\item \hyperlink{tut-xo_remotepbs}{Using a remote P\+B\+S cluster}
\item \hyperlink{tut-xo_remotesge}{Using a remote S\+G\+E cluster}
\item \hyperlink{tut-xo_remoteslurm}{Using a remote S\+L\+U\+R\+M cluster}
\item \hyperlink{tut-xo_localqi}{Running optimations locally}
\end{DoxyItemize}
\item \hyperlink{tut-xo_files}{What is written to the local directory?}
\item \hyperlink{tut-xo_search-set}{Search Settings}
\item \hyperlink{tut-xo_begin}{\char`\"{}\+Begin\char`\"{}}
\item \hyperlink{tut-xo_prog-mon}{Monitor progress}
\begin{DoxyItemize}
\item \hyperlink{tut-xo_trends}{View trends}
\end{DoxyItemize}
\item \hyperlink{tut-xo_xo_tut_further}{Further reading}
\begin{DoxyItemize}
\item \hyperlink{optschemes}{Optimization Schemes}
\item \hyperlink{xo_saveresume}{Saving and Resuming Sessions in Xtal\+Opt}
\end{DoxyItemize}
\end{DoxyItemize}\hypertarget{tut-xo_launch}{}\section{Launch Xtal\+Opt}\label{tut-xo_launch}
Open avogadro, go to the \char`\"{}\+Extensions\char`\"{} menu and select \char`\"{}\+Xtal\+Opt\char`\"{}.\hypertarget{tut-xo_init}{}\section{Enter composition and restraints}\label{tut-xo_init}
 
\begin{DoxyImageNoCaption}
  \mbox{\includegraphics[width=\textwidth]{struct-lim.png}}
\end{DoxyImageNoCaption}


The interface opens to the \char`\"{}\+Structure Limits\char`\"{} tab, shown above. We will use a 6 formula unit supercell of titanium dioxide for this tutorial, so enter \char`\"{}\+Ti6 O12\char`\"{} for the cell composition. We will assume that we know nothing about the system and use very loose restraints. Set all cell length minima to 1 angstrom and maxima to 20 angstrom. Constrain the angles to be between 60 and 120 degrees, and the volume from between 1 and 500 cubic angstrom. Specify a minimum interatomic distance of 0.\+5 angstrom. (Note that due to the angle adjustment described in the C\+P\+C 2010 publication, 60-\/120 degrees is the largest range of cell angles that Xtal\+Opt will generate.)\hypertarget{tut-xo_opt}{}\section{Optimizer setup}\label{tut-xo_opt}
Xtal\+Opt currently supports the \hyperlink{tut-xo_vasp-opt}{V\+A\+S\+P}, \hyperlink{tut-xo_gulp-opt}{G\+U\+L\+P}, \hyperlink{tut-xo_pwscf-opt}{P\+Wscf}, and \hyperlink{tut-xo_castep-opt}{C\+A\+S\+T\+E\+P} codes for performing geometry optimizations. Each is detailed in its own section below.\hypertarget{tut-xo_vasp-opt}{}\subsection{V\+A\+S\+P}\label{tut-xo_vasp-opt}
 
\begin{DoxyImageNoCaption}
  \mbox{\includegraphics[width=\textwidth]{opt-set-vasp.png}}
\end{DoxyImageNoCaption}


On the next tab, load the optimization scheme by clicking the \char`\"{}\+Load
\+Opt Scheme\char`\"{} button and selecting the \char`\"{}samples/vasp-\/xtalopt.\+scheme\char`\"{} file that is distributed with the source code. If you do not have a copy of the source code, the scheme file can be obtained by clicking the \char`\"{}\+Original Format\char`\"{} link at the bottom of the page \href{http://xtalopt.openmolecules.net/wiki/index.fcgi/browser/samples/vasp-xtalopt.scheme}{\tt here}.

For more details on optimization schemes, see \hyperlink{optschemes}{Optimization Schemes}.

After loading the optimization scheme, Xtal\+Opt will prompt for the P\+O\+T\+C\+A\+R files to use. Select files appropriated for the prompted atom. Xtal\+Opt will construct the P\+O\+T\+C\+A\+R files on the local computer, and then copy them over to the cluster when the calculation is submitted. It is necessary to have the V\+A\+S\+P P\+O\+T\+C\+A\+R files for each atomic species located somewhere on the local computer. See the V\+A\+S\+P manual for information on obtaining the P\+O\+T\+C\+A\+R files.

Take a moment to look through each file for each optimization step. Notice that the I\+N\+C\+A\+R template includes two user-\/specified values, \%user2\% and \%user3\% for the external pressure and the energy cutoff, respectively. By entering appropriate values in the \char`\"{}user2\+:\char`\"{} and \char`\"{}user3\+:\char`\"{} fields on the left, it is easy to update these values for all optimization steps.

Notice the other \%keyword\% values in the job.\+pbs templates. These are used to enter information that is specific to a search or structure when the actual input files are written prior to job submission. Click the \char`\"{}\+Help\char`\"{} button for a full listing of the available keywords.

Xtal\+Opt expects V\+A\+S\+P to use the default filenames, mainly P\+O\+S\+C\+A\+R, C\+O\+N\+T\+C\+A\+R, and O\+U\+T\+C\+A\+R.

\hyperlink{tut-xo_qisetup}{Skip to next section.}\hypertarget{tut-xo_gulp-opt}{}\subsection{G\+U\+L\+P}\label{tut-xo_gulp-opt}
 
\begin{DoxyImageNoCaption}
  \mbox{\includegraphics[width=\textwidth]{opt-set-gulp.png}}
\end{DoxyImageNoCaption}


On the next tab we choose G\+U\+L\+P for the local optimizer and enter a template for G\+U\+L\+P to use. Select \char`\"{}\+G\+U\+L\+P\char`\"{} as the \char`\"{}\+Optimizer\char`\"{} and \char`\"{}xtal.\+gin\char`\"{} as \char`\"{}\+Template\char`\"{}. Next, fill out the text field on the right with the following template\+: 
\begin{DoxyCode}
opti conj conp
switch\_minimiser bfgs gnorm 0.5
cell
  %a% %b% %c% %alphaDeg% %betaDeg% %gammaDeg%
frac
%coordsFrac%
species
Ti 2.196
O -1.098
buck
Ti Ti 31120.1 0.1540 5.25  15
O  O  11782.7 0.2340 30.22 15
Ti O  16957.5 0.1940 12.59 15
lennard 12 6
Ti Ti 1 0 15
O  O  1 0 15
Ti O  1 0 15
\end{DoxyCode}


Alternatively, one can load the scheme file distributed with the source code under samples/gulp-\/\+Ti\+O-\/xtalopt.\+scheme. If the source code is not available, the scheme file can be obtained by clicking the \char`\"{}\+Original Format\char`\"{} link at the bottom of the page \href{http://xtalopt.openmolecules.net/wiki/index.fcgi/browser/samples/gulp-TiO-xtalopt.scheme}{\tt here}.

For more details on optimization schemes, see \hyperlink{optschemes}{Optimization Schemes}.

Note the \char`\"{}\%\char`\"{} surrounding various keywords. These will be replaced by the structure-\/specific data when the optimizer is invoked for each structure. Click \char`\"{}\+Help\char`\"{} to view all of the keywords available. The number of optimization steps can be modified with the \char`\"{}\+Add/\+Resume\char`\"{} buttons. The \char`\"{}user\char`\"{} fields in the lower left corner allow users to specify their own keyword/value pairs, which is useful for making changes to multiple optimization steps at once. We will only be using one optimization step in this tutorial.

Xtal\+Opt expects G\+U\+L\+P to use the following filenames\+:


\begin{DoxyCode}
gulp < xtal.gin > xtal.got
\end{DoxyCode}


\hyperlink{tut-xo_qisetup}{Skip to next section.}\hypertarget{tut-xo_pwscf-opt}{}\subsection{P\+Wscf}\label{tut-xo_pwscf-opt}
 
\begin{DoxyImageNoCaption}
  \mbox{\includegraphics[width=\textwidth]{opt-set-pwscf.png}}
\end{DoxyImageNoCaption}


On the next tab, load the optimization scheme that is distributed with the source code under the samples/ directory. The scheme that we want is named \char`\"{}pwscf-\/xtalopt.\+scheme\char`\"{}. If the source code is not available, the scheme file can be obtained by clicking the \char`\"{}\+Original Format\char`\"{} link at the bottom of the page \href{http://xtalopt.openmolecules.net/wiki/index.fcgi/browser/samples/pwscf-xtalopt.scheme}{\tt here}.

For more details on optimization schemes, see \hyperlink{optschemes}{Optimization Schemes}.

Each P\+Wscf input file will need to be edited to specify\+:
\begin{DoxyEnumerate}
\item The pseudo\+\_\+dir containing the pseudopotential files on the remote cluster, and
\item The pseudopotentials for each atom (under A\+T\+O\+M\+I\+C\+\_\+\+S\+P\+E\+C\+I\+E\+S)
\end{DoxyEnumerate}

Take a moment to look through each file for each optimization step.

Notice the \%keyword\% values in the job.\+pbs templates. These are used to enter information that is specific to a search or structure when the actual input files are written prior to job submission. Click the \char`\"{}\+Help\char`\"{} button for a full listing of the available keywords.

Be aware that every P\+Wscf/\+C\+A\+S\+T\+E\+P installation is different, and it is almost certain that the job.\+pbs file included with this scheme will not work on any cluster other than the Zurek group\textquotesingle{}s \char`\"{}parity\char`\"{} cluster at S\+U\+N\+Y Buffalo\textquotesingle{}s Center for Computational Resources. It may take some experimentation to get jobs to submit successfully, and you may need to contact the mantainers of the cluster for assistance for information about M\+P\+I, executable locations, etc. Perhaps the easiest method to find the correct P\+B\+S script is to run some trial submissions by hand, and then replace the structure/search specific information with the appropriate keywords once a working script has been generated.

Xtal\+Opt expects P\+Wscf to use the following filenames\+:


\begin{DoxyCode}
pw.x < xtal.in > xtal.out
\end{DoxyCode}


\hyperlink{tut-xo_qisetup}{Skip to next section.}\hypertarget{tut-xo_castep-opt}{}\subsection{C\+A\+S\+T\+E\+P}\label{tut-xo_castep-opt}
 
\begin{DoxyImageNoCaption}
  \mbox{\includegraphics[width=\textwidth]{opt-set-castep.png}}
\end{DoxyImageNoCaption}


On the next tab, load the optimization scheme that is distributed with the source code under the samples/ directory. The scheme that we want is named \char`\"{}castep-\/xtalopt.\+scheme\char`\"{}. If the source code is not available, the scheme file can be obtained by clicking the \char`\"{}\+Original Format\char`\"{} link at the bottom of the page \href{http://xtalopt.openmolecules.net/wiki/index.fcgi/browser/samples/castep-xtalopt.scheme}{\tt here}.

For more details on optimization schemes, see \hyperlink{optschemes}{Optimization Schemes}.

It is important to note that C\+A\+S\+T\+E\+P input files require the \char`\"{}\%\char`\"{} character to define blocks. The percent character is special in the Xtal\+Opt input template parser to define keywords (see below). To insert a literal \char`\"{}\%\char`\"{} into the input, use percent\%.

E.\+g. Specification of the fractional coordinate block in the .cell template should look like\+:


\begin{DoxyCode}
%percent%BLOCK POSITIONS\_FRAC
%coordsFrac%
%percent%ENDBLOCK POSITIONS\_FRAC
\end{DoxyCode}


Take a moment to look through each file for each optimization step.

Notice the \%keyword\% values in the job.\+pbs templates. These are used to enter information that is specific to a search or structure when the actual input files are written prior to job submission. Click the \char`\"{}\+Help\char`\"{} button for a full listing of the available keywords.

Be aware that every P\+Wscf/\+C\+A\+S\+T\+E\+P installation is different, and it is almost certain that the job.\+pbs file included with this scheme will not work on any cluster other than the Zurek group\textquotesingle{}s \char`\"{}parity\char`\"{} cluster at S\+U\+N\+Y Buffalo\textquotesingle{}s Center for Computational Resources. It may take some experimentation to get jobs to submit successfully, and you may need to contact the mantainers of the cluster for assistance for information about M\+P\+I, executable locations, etc. Perhaps the easiest method to find the correct P\+B\+S script is to run some trial submissions by hand, and then replace the structure/search specific information with the appropriate keywords once a working script has been generated.

Xtal\+Opt expects C\+A\+S\+T\+E\+P to use the following filenames\+:


\begin{DoxyCode}
\textcolor{preprocessor}{# XtalOpt will write xtal.cell, xtal.param}
castep xtal
\textcolor{preprocessor}{# CASTEP will create xtal.castep}
\end{DoxyCode}
\hypertarget{tut-xo_qisetup}{}\section{Queue setup}\label{tut-xo_qisetup}
Xtal\+Opt currently supports using the \hyperlink{tut-xo_remotepbs}{P\+B\+S}, \hyperlink{tut-xo_remotesge}{S\+G\+E}, and \hyperlink{tut-xo_remoteslurm}{S\+L\+U\+R\+M} queuing systems on remote S\+S\+H-\/accessible clusters, as well as an internal \hyperlink{tut-xo_localqi}{local} queue that manages calculations on the user\textquotesingle{}s workstation. Each queueing interface is detailed in its own section below.\hypertarget{tut-xo_remotepbs}{}\subsection{Using a remote P\+B\+S cluster}\label{tut-xo_remotepbs}
 
\begin{DoxyImageNoCaption}
  \mbox{\includegraphics[width=\textwidth]{opt-set-pbs.png}}
\end{DoxyImageNoCaption}


Select \char`\"{}\+P\+B\+S\char`\"{} from the list of Queues, and then click the \char`\"{}\+Configure...\char`\"{} button. A new window will prompt for\+:
\begin{DoxyItemize}
\item host\+: The hostname of the P\+B\+S cluster\textquotesingle{}s head node
\item user\+: The username used to log into the cluster
\item Working directory (Server)\+: A directory that is readable/writable by \char`\"{}user\char`\"{} on the cluster, used when performing optimizations.
\item Working directory (Local)\+: A directory that is readable/writable by the current user on the local computer. This is where the final structures and resume files are written.
\item Description\+: Used for the \%description\% keyword in input templates.
\item Path to qsub\+: Where to find the qsub executable on the remote cluster. Note that if qsub is in the cluster\textquotesingle{}s \$\+P\+A\+T\+H, setting this to just \textquotesingle{}qsub\textquotesingle{} will work.
\item Path to qdel\+: Where to find the qdel executable on the remote cluster. Note that if qdel is in the cluster\textquotesingle{}s \$\+P\+A\+T\+H, setting this to just \textquotesingle{}qdel\textquotesingle{} will work.
\item Path to qstat\+: Where to find the qstat executable on the remote cluster. Note that if qstat is in the cluster\textquotesingle{}s \$\+P\+A\+T\+H, setting this to just \textquotesingle{}qstat\textquotesingle{} will work.
\end{DoxyItemize}

A new template, \char`\"{}job.\+pbs\char`\"{} is added to the list of available templates. This is the job submission script for P\+B\+S. This script should roughly follow this design\+:


\begin{DoxyCode}
\textcolor{preprocessor}{#/bin/bash}
\textcolor{preprocessor}{#PBS -l nodes=1:ppn=8}
\textcolor{preprocessor}{#PBS -o ../%gen%x%id%-%optstep%.out}
\textcolor{preprocessor}{#PBS -e ../%gen%x%id%-%optstep%.err}
\textcolor{preprocessor}{#PBS -N %description%-%gen%x%id%-%optstep%}

\textcolor{preprocessor}{###Include this for XtalOpt scripts!###}
export PBS\_O\_WORKDIR=%rempath%

\textcolor{preprocessor}{# Change to structure's working directory, copy input files to node's scratch dirs:}
\textcolor{keywordflow}{for} node in `cat $PBS\_NODEFILE | sort | uniq`; \textcolor{keywordflow}{do}
rsh $node \textcolor{stringliteral}{"cp $PBS\_O\_WORKDIR/* $PBSTMPDIR/;"};
done

\textcolor{preprocessor}{# Move to the scratch directory}
cd $PBSTMPDIR
echo \textcolor{stringliteral}{"running in directory $PBSTMPDIR"}

\textcolor{preprocessor}{# Set any environment variables needed for the optimizer/MPI here:}

\textcolor{preprocessor}{# Run optimizer, be sure to use the filenames that XtalOpt expects.}
\textcolor{preprocessor}{# See the template menu in XtalOpt and the example templates in the}
\textcolor{preprocessor}{# samples/ directory of the XtalOpt sources.}

\textcolor{preprocessor}{# Don't forget to clean up after MPI if needed!}

\textcolor{comment}{// Print files from each node}
\textcolor{keywordflow}{for} node in `cat $PBS\_NODEFILE | sort | uniq`; \textcolor{keywordflow}{do}
echo \textcolor{stringliteral}{"$node:"}
rsh $node \textcolor{stringliteral}{"ls -l $PBSTMPDIR"}
done
\textcolor{preprocessor}{# Copy back results from master node's scratch directory}
\textcolor{preprocessor}{cp $PBSTMPDIR}\textcolor{comment}{/* $PBS\_O\_WORKDIR/}
\end{DoxyCode}


For more details on optimization schemes, see \hyperlink{optschemes}{Optimization Schemes}.

Be aware that every installation is different, and it is almost certain that the job.\+pbs file included with this scheme will not work on any cluster other than the Zurek group\textquotesingle{}s \char`\"{}parity\char`\"{} cluster at S\+U\+N\+Y Buffalo\textquotesingle{}s Center for Computational Resources. It may take some experimentation to get jobs to submit successfully, and you may need to contact the mantainers of the cluster for assistance or information about M\+P\+I, executable locations, etc. Perhaps the easiest method to find the correct P\+B\+S script is to run some trial submissions by hand, and then replace the structure/search specific information with the appropriate keywords once a working script has been generated.

A handy trick for monitoring jobs outside of Xtal\+Opt is to include the following line in job.\+pbs\+:


\begin{DoxyCode}
\textcolor{preprocessor}{#PBS -N %description%-%gen%x%id%-%optstep%}
\end{DoxyCode}


This will name each job, for example, xtal\+Search-\/3x4-\/2, where xtal\+Search is a user-\/specified description of the search, and 3x4-\/2 means that it is the fourth structure in the third generation running its second optimization step.

\hyperlink{tut-xo_files}{Skip to next section.}\hypertarget{tut-xo_remotesge}{}\subsection{Using a remote S\+G\+E cluster}\label{tut-xo_remotesge}
 
\begin{DoxyImageNoCaption}
  \mbox{\includegraphics[width=\textwidth]{opt-set-pbs.png}}
\end{DoxyImageNoCaption}
 \begin{DoxyRefDesc}{Todo}
\item[\hyperlink{todo__todo000001}{Todo}]Get screenshots of S\+G\+E config dialog\end{DoxyRefDesc}


Select \char`\"{}\+S\+G\+E\char`\"{} from the list of Queues, and then click the \char`\"{}\+Configure...\char`\"{} button. A new window will prompt for\+:
\begin{DoxyItemize}
\item host\+: The hostname of the S\+G\+E cluster\textquotesingle{}s head node
\item user\+: The username used to log into the cluster
\item Working directory (Server)\+: A directory that is readable/writable by \char`\"{}user\char`\"{} on the cluster, used when performing optimizations.
\item Working directory (Local)\+: A directory that is readable/writable by the current user on the local computer. This is where the final structures and resume files are written.
\item Description\+: Used for the \%description\% keyword in input templates.
\item Path to qsub\+: Where to find the qsub executable on the remote cluster. Note that if qsub is in the cluster\textquotesingle{}s \$\+P\+A\+T\+H, setting this to just \textquotesingle{}qsub\textquotesingle{} will work.
\item Path to qdel\+: Where to find the qdel executable on the remote cluster. Note that if qdel is in the cluster\textquotesingle{}s \$\+P\+A\+T\+H, setting this to just \textquotesingle{}qdel\textquotesingle{} will work.
\item Path to qstat\+: Where to find the qstat executable on the remote cluster. Note that if qstat is in the cluster\textquotesingle{}s \$\+P\+A\+T\+H, setting this to just \textquotesingle{}qstat\textquotesingle{} will work.
\end{DoxyItemize}

\begin{DoxyRefDesc}{Todo}
\item[\hyperlink{todo__todo000002}{Todo}]Get template for job.\+sge scripts\end{DoxyRefDesc}


A new template, \char`\"{}job.\+sge\char`\"{} is added to the list of available templates. This is the job submission script for S\+G\+E. It may take some experimentation to get jobs to submit successfully, and you may need to contact the mantainers of the cluster for assistance or information about M\+P\+I, executable locations, etc. Perhaps the easiest method to find the correct S\+G\+E script is to run some trial submissions by hand, and then replace the structure/search specific information with the appropriate keywords once a working script has been generated.

For more details on optimization schemes, see \hyperlink{optschemes}{Optimization Schemes}.

\hyperlink{tut-xo_files}{Skip to next section.}\hypertarget{tut-xo_remoteslurm}{}\subsection{Using a remote S\+L\+U\+R\+M cluster}\label{tut-xo_remoteslurm}
 
\begin{DoxyImageNoCaption}
  \mbox{\includegraphics[width=\textwidth]{opt-set-slurm.png}}
\end{DoxyImageNoCaption}


Select \char`\"{}\+S\+L\+U\+R\+M\char`\"{} from the list of Queues, and then click the \char`\"{}\+Configure...\char`\"{} button. A new window will prompt for\+:
\begin{DoxyItemize}
\item host\+: The hostname of the S\+G\+E cluster\textquotesingle{}s head node
\item user\+: The username used to log into the cluster
\item Working directory (Server)\+: A directory that is readable/writable by \char`\"{}user\char`\"{} on the cluster, used when performing optimizations.
\item Working directory (Local)\+: A directory that is readable/writable by the current user on the local computer. This is where the final structures and resume files are written.
\item Description\+: Used for the \%description\% keyword in input templates.
\item Path to sbatch\+: Where to find the sbatch executable on the remote cluster. Note that if sbatch is in the cluster\textquotesingle{}s \$\+P\+A\+T\+H, setting this to just \textquotesingle{}sbatch\textquotesingle{} will work.
\item Path to scancel\+: Where to find the scancel executable on the remote cluster. Note that if scancel is in the cluster\textquotesingle{}s \$\+P\+A\+T\+H, setting this to just \textquotesingle{}scancel\textquotesingle{} will work.
\item Path to squeue\+: Where to find the squeue executable on the remote cluster. Note that if squeue is in the cluster\textquotesingle{}s \$\+P\+A\+T\+H, setting this to just \textquotesingle{}squeue\textquotesingle{} will work.
\end{DoxyItemize}

\begin{DoxyRefDesc}{Todo}
\item[\hyperlink{todo__todo000003}{Todo}]Get template for job.\+slurm scripts\end{DoxyRefDesc}


A new template, \char`\"{}job.\+slurm\char`\"{} is added to the list of available templates. This is the job submission script for S\+L\+U\+R\+M. It may take some experimentation to get jobs to submit successfully, and you may need to contact the mantainers of the cluster for assistance or information about M\+P\+I, executable locations, etc. Perhaps the easiest method to find the correct S\+L\+U\+R\+M script is to run some trial submissions by hand, and then replace the structure/search specific information with the appropriate keywords once a working script has been generated.

For more details on optimization schemes, see \hyperlink{optschemes}{Optimization Schemes}.

\hyperlink{tut-xo_files}{Skip to next section.}\hypertarget{tut-xo_localqi}{}\subsection{Running optimations locally}\label{tut-xo_localqi}
 
\begin{DoxyImageNoCaption}
  \mbox{\includegraphics[width=\textwidth]{opt-set-local.png}}
\end{DoxyImageNoCaption}


Select \char`\"{}\+Local\char`\"{} from the list of Queues, and then click the configure button. A new window will prompt for\+:
\begin{DoxyItemize}
\item Local working directory\+: A directory that is readable/writable by the current user on the local computer. This is where the final structures and resume files are written.
\end{DoxyItemize}

If the optimizer\textquotesingle{}s executable (vasp, gulp, pw.\+x, castep, etc) is not in your system path, you will need to specify the location of the executable by clicking the \char`\"{}\+Configure...\char`\"{} button next to the optimizer selection menu.

For more details on optimization schemes, see \hyperlink{optschemes}{Optimization Schemes}.\hypertarget{tut-xo_files}{}\section{What is written to the local directory?}\label{tut-xo_files}
A directory for each structure is created at


\begin{DoxyCode}
[Local working directory]/<gen#>x<id#>
\end{DoxyCode}


that will contain input, output, and data files specific to each structure. Two additional files are also written to the local filesystem\+:


\begin{DoxyCode}
[Local working directory]/xtalopt.state
\end{DoxyCode}


which contains save/resume information to continue a session that has been stopped, and


\begin{DoxyCode}
[Local working directory]/results.txt
\end{DoxyCode}


which stores a list of all structures sorted by increasing enthalpy. The latter file is handy for offline analysis, since there is no need to open Xtal\+Opt to find the most stable structures of a previous search.\hypertarget{tut-xo_search-set}{}\section{Search Settings}\label{tut-xo_search-set}
 
\begin{DoxyImageNoCaption}
  \mbox{\includegraphics[width=\textwidth]{search-set.png}}
\end{DoxyImageNoCaption}


In the \char`\"{}\+Search Settings\char`\"{} tab, most of the default settings should suffice (See C\+P\+C 2010 publication). We arbitrarily set the initial structures to 20 and the continuous structures to 5, although these may need to be adjusted based on available resources. We will not specify initial seeds, but the option to do so exists on this screen.

It is not neccessary to limit the number of running jobs unless running locally, as the P\+B\+S queue on the cluster will manage job control for us. If running locally, set the job limit no higher than \mbox{[}number of available processor cores\mbox{]} -\/ 1 (e.\+g. for a quadcore processor, allow three jobs to run simultaneously). This allows one core to remain free for the system to run.\hypertarget{tut-xo_begin}{}\section{\char`\"{}\+Begin\char`\"{}}\label{tut-xo_begin}

\begin{DoxyImage}
\includegraphics[width=\textwidth]{prog-start.png}
\caption{The ``\+Progress\textquotesingle{}\textquotesingle{} tab immediately after starting a search}
\end{DoxyImage}


Xtal\+Opt has everything it needs to start its search at this point; click the \char`\"{}\+Begin\char`\"{} button in the lower right corner of the application to tell it to start the search algorithm. A progress bar appears as the random first generation is created. Switch to the \char`\"{}\+Progress\char`\"{} tab and 20 entries will appear, all with a status of \char`\"{}\+Waiting for
\+Optimization\char`\"{}. Click \char`\"{}\+Refresh\char`\"{} on this tab to begin the local optimizations. From here, Xtal\+Opt will continue to run without user input, starting new optimizations and generating new structures until it is stopped by the user.\hypertarget{tut-xo_prog-mon}{}\section{Monitor progress}\label{tut-xo_prog-mon}

\begin{DoxyImage}
\includegraphics[width=\textwidth]{prog-mon.png}
\caption{The ``\+Progress\textquotesingle{}\textquotesingle{} tab mid-\/run}
\end{DoxyImage}


As Xtal\+Opt performs the search, the progress table continuously updates, providing information about each structure. We see individuals in various stages of completion\+: most are optimized (in yellow), structure 2x7 has been automatically marked as a duplicate (dark green) of structure 3x3 and removed from the breeding pool, structure 4x4 is currently undergoing a local optimization (light green), while structure 4x5 is waiting to be optimized (blue).

Other useful information is displayed about each structure, such as the time spent in optimization, the optimized enthalpy, the cell volume, spacegroup, and each structure\textquotesingle{}s ancestory (i.\+e. parent(s) and parameters for the genetic operator that generated it). A status bar on the bottom of the window shows the number of structures that are optimized, running, and failing at any given time. This information is visible regardless of which tab is currently being viewed.

An additional feature of the progress table is the ability to immediately visualize any of the individuals in the Avogadro main window -- simply clicking on a row in this table will display the three-\/dimensional structure in Avogadro, where it can be visualized, modified, or exported. If the user would like to add a bit of \char`\"{}intelligent design\char`\"{} to the evolutionary process, a structure can be modified and then resubmitted using a context (right-\/click) menu from the progress table. The context menu provides tools to (un)kill a structure, resubmit for local optimization at an arbitrary optimization step, or replace a problematic structure with a new, random individual.\hypertarget{tut-xo_trends}{}\subsection{View trends}\label{tut-xo_trends}

\begin{DoxyImage}
\includegraphics[width=\textwidth]{trend-view.png}
\caption{The ``\+Plot\textquotesingle{}\textquotesingle{} tab mid-\/run displaying enthalpy vs. volume. Each structure is labeled with its Hermann Mauguin spacegroup symbol.}
\end{DoxyImage}


Another visualization and analysis tool available during the search is the interactive plot. The plot is capable of investigating trends in the search by plotting a point for each individual using structure number, generation number, enthalpy, energy, \$\+P\+V\$ enthalpy term, lattice parameters, or cell volume on either axis. This powerful feature allows the user to visualize complex relationships present in the generated structures. E.\+g., a plot of enthalpy vs. structure number provides an overview of the search\textquotesingle{}s progress. Or, recalling that H = U + P\+V, plotting enthalpy vs. P\+V enthalpy term or energy lends insight into whether the enthalpy (H) is dominated by atomic interactions (U) or cell parameters (P\+V). Further information is available by labeling the points with the individual\textquotesingle{}s spacegroup number, Hermann Mauguin spacegroup symbol, enthalpy, energy, P\+V term, volume, generation, or index number.

A particularly useful plot is that of enthalpy vs. cell volume, as shown above. From this view, we see a general trend that enthalpy increases with volume (the effect is much more pronounced for systems at higher pressures), and also that below a certain volume enthalpy rises sharply. From this data set, we see that there is a cluster of very low enthalpy structures with cell volumes around 180 cubic angstroms. Armed with this data, we can update the starting volume on the Cell Initialization tab mid-\/run to reflect this new piece of information that the search has provided us. Many of the other parameters governing structure generation and algorithm specifics can be similarly modified during a search without the need to restart the algorithm.

The plot is also interactive; zooming and panning are possible using simple mouse controls. Clicking on a structure\textquotesingle{}s point on the plot will load it in the main Avogadro window, allow all the same functionality as described above in \hyperlink{tut-xo_prog-mon}{Monitor progress}.\hypertarget{tut-xo_xo_tut_further}{}\section{Further reading}\label{tut-xo_xo_tut_further}
From here, you may be interested in\+:
\begin{DoxyItemize}
\item \hyperlink{optschemes}{Optimization Schemes}
\item \hyperlink{xo_saveresume}{Saving and Resuming Sessions in Xtal\+Opt} 
\end{DoxyItemize}